\documentclass{beamer}

% This file is a solution template for:

% - Giving a talk on some subject.
% - The talk is between 15min and 45min long.
% - Style is ornate.



% Copyright 2004 by Till Tantau <tantau@users.sourceforge.net>.
%
% In principle, this file can be redistributed and/or modified under
% the terms of the GNU Public License, version 2.
%
% However, this file is supposed to be a template to be modified
% for your own needs. For this reason, if you use this file as a
% template and not specifically distribute it as part of a another
% package/program, I grant the extra permission to freely copy and
% modify this file as you see fit and even to delete this copyright
% notice. 

\newcommand{\code}[1]{\texttt{#1}}

\newcommand{\startor}{\left\{ \begin{array}{ll}}
\newcommand{\stopor}{\end{array} \right.}
\newcommand{\otherwise}{\textrm{otherwise}}

\newcommand{\green}[1]{\textcolor{green}{#1}}
\newcommand{\red}[1]{\textcolor{red}{#1}}
\newcommand{\blue}[1]{\textcolor{blue}{#1}}

\mode<presentation>
{
  \usetheme{Warsaw}
  % or ...

  \setbeamercovered{transparent}
  % or whatever (possibly just delete it)
}


\usepackage[english]{babel}
% or whatever

\usepackage[latin1]{inputenc}
% or whatever

\usepackage{times}
\usepackage[T1]{fontenc}
% Or whatever. Note that the encoding and the font should match. If T1
% does not look nice, try deleting the line with the fontenc.


\title[Probabilistic Scheme] % (optional, use only with long paper titles)
{Report on the Probabilistic Language Scheme}

%\subtitle
%{Presentation Subtitle} % (optional)

\author{Alexey Radul}
% - Use the \inst{?} command only if the authors have different
%   affiliation.

\institute[MIT] % (optional, but mostly needed)
{
  Computer Science and Artificial Intelligence Laboratory\\
  Massachusetts Institute of Technology
}
% - Use the \inst command only if there are several affiliations.
% - Keep it simple, no one is interested in your street address.

\date{DLS 2007, Oct 22nd}

\subject{Probability, Scheme}
% This is only inserted into the PDF information catalog. Can be left
% out. 



% If you have a file called "university-logo-filename.xxx", where xxx
% is a graphic format that can be processed by latex or pdflatex,
% resp., then you can add a logo as follows:

% \pgfdeclareimage[height=0.5cm]{university-logo}{university-logo-filename}
% \logo{\pgfuseimage{university-logo}}



% Delete this, if you do not want the table of contents to pop up at
% the beginning of each subsection:
%% \AtBeginSection[]
%% {
%%   \begin{frame}<beamer>
%%     \frametitle{Outline}
%%     \tableofcontents[currentsection]
%%   \end{frame}
%% }


% If you wish to uncover everything in a step-wise fashion, uncomment
% the following command: 

%\beamerdefaultoverlayspecification{<+->}


\begin{document}

\begin{frame}
  \titlepage
\end{frame}

\begin{frame}
  \frametitle{Outline}
  \tableofcontents
  % You might wish to add the option [pausesections]
\end{frame}


% Since this a solution template for a generic talk, very little can
% be said about how it should be structured. However, the talk length
% of between 15min and 45min and the theme suggest that you stick to
% the following rules:  

% - Exactly two or three sections (other than the summary).
% - At *most* three subsections per section.
% - Talk about 30s to 2min per frame. So there should be between about
%   15 and 30 frames, all told.

\section{Motivation}

\subsection{Background}

\begin{frame}

\begin{center}
\Huge Probability theory exists:
\end{center}

  \[ p(A \textrm{ and } B) = p(A) * p(B|A) = p(B) * p(A|B) \]

  \[ p(B|A) = p(B) * p(A|B) / p(A) \]

\end{frame}

\begin{frame}

  {\huge
    Probabilistic inference is useful
  }

  \begin{itemize}
  \item for spam filtering, Sahami et al 1998
  \item for robots driving through deserts, Thrun et al 2006
  \item for studying gene expression, Segal et al 2001
%  \only<2->{ \item for publishing papers, Radul 2007 }
  \item and many, many more
  \end{itemize}
\end{frame}

\subsection{The Problem}

\begin{frame}
  
  {\huge \ldots but hard to use}

  \begin{itemize}
  \item algorithms are complicated
  \item existing systems are a pain to use
    \begin{itemize}
    \item hew close to their assumptions
    \item not modular
    \item hard to interoperate with
    \end{itemize}
  \end{itemize}
\end{frame}

\begin{frame}
\begin{center}
\Huge Can we do better?
\end{center}
\end{frame}

\subsection{The Approach}

\begin{frame}

\vspace{0.40in}
\begin{center}
\Huge We can try
\end{center}

\pause
  \begin{itemize}
  \item library for Scheme
  \item experiment in language design
  \end{itemize}
\end{frame}

\section{Representation}

\begin{frame}
\begin{center}
\Huge First big question:
\end{center}
\end{frame}

\begin{frame}
\begin{center}
\Huge Representation?
\end{center}
\end{frame}

\subsection{Lists?}

\begin{frame}
\begin{center}
\Huge Lists?
\end{center}
\end{frame}

\begin{frame}
\begin{center}
\Huge Lists lose on long tails
\end{center}
\end{frame}

\begin{frame}

  {\huge Long tails}

  \begin{itemize}
  \item possible parse trees of a sentence
    \begin{itemize}
    \item There are a vast number of them, but most are extremely unlikely
    \end{itemize}
  \item how many times will one flip heads on a fair coin before the first tail?
    \begin{itemize}
    \item Infinite, but again, the tail is probably irrelevant
    \end{itemize}
  \item and many, many more
  \end{itemize}
\end{frame}

\begin{frame}
\begin{center}
\Huge So?
\end{center}
\end{frame}

\subsection{Lazy Streams}

\begin{frame}
\begin{center}
\Huge Lazy Streams
\end{center}
\end{frame}

\begin{frame}
\begin{center}
\Huge Lazy Streams
\end{center}
\begin{itemize}
\item Delay computing the long tail
\item You likely won't need it anyway
\end{itemize}
\end{frame}

\begin{frame}
\begin{center}
\Huge Approximation,
\end{center}
\pause
but the best kind:
\begin{itemize}
\item anytime
\item restartable
\item bounded-error
\end{itemize}
\end{frame}

\begin{frame}
\begin{center}
\Huge There's also some fine print
\end{center}
\tiny
The streams need to allow duplicates. \\
The streams need to allow explicit statements of impossibility. \\
The objects exiting the streams need to be cached. \\
The caches need to be kept up to date \\
Even in the face of aliasing and direct access to the streams. \\
If you really want to know, ask during the question period.

\only<-1>{ \vspace{0.733in} }
\only<2->{
\begin{center}
\Huge But it all works out
\end{center} }
\end{frame}

\section{Interface}

\begin{frame}
\begin{center}
\Huge Second big question:
\end{center}
\end{frame}

\begin{frame}
\begin{center}
\Huge API?
\end{center}
\end{frame}

\subsection{Explicit Distributions?}

\begin{frame}
\begin{center}
\Huge Explicit Distribution Objects?
\end{center}
\end{frame}

\begin{frame}

  \[ p(\green{f}(x, y)) = \sum_{x', y' \textrm{ with } \green{f}(x', y') = \green{f}(x, y)} \red{p(x')} * \blue{p(y'|x')} \]
  \code{(dependent-product \red{distribution} \blue{conditional} \green{combiner})}

\[ p(x|\blue{A(x)}) = \startor
\red{p(x)} / p(\blue{A}) & \textrm{if $\blue{A(x)}$ is true} \\
0 &                        \textrm{if $\blue{A(x)}$ is false} \\
\stopor \]
  \code{(conditional-distribution \red{distribution} \blue{predicate})}

\end{frame}

\begin{frame}
\begin{center}
\Huge Looks ok, but...
\end{center}
\end{frame}

\begin{frame}[fragile=singleslide]
\begin{verbatim}
(define die-roll-distribution
  (make-discrete-distribution
   '(1 1/6) '(2 1/6) '(3 1/6)
   '(4 1/6) '(5 1/6) '(6 1/6)))

(let ((two-die-roll-distribution
       (dependent-product
        die-roll-distribution
        (lambda (result1) die-roll-distribution)
        +)))
  (conditional-distribution
   two-die-roll-distribution
   (lambda (sum) (> sum 9))))
\end{verbatim}
\end{frame}

\begin{frame}
\begin{center}
\Huge Instead:
\end{center}
\end{frame}

\subsection{Implicit Distributions?}

\begin{frame}[fragile=singleslide]
\begin{verbatim}
(define (roll-die)
  (discrete-select
   (1 1/6) (2 1/6) (3 1/6)
   (4 1/6) (5 1/6) (6 1/6)))

(let ((num (+ (roll-die) (roll-die))))
  (observe! (> num 9))
  num)
\end{verbatim}
\begin{center}
\Huge Implicit Distributions?
\end{center}
\end{frame}

\begin{frame}[fragile=singleslide]
\begin{verbatim}
(define (roll-die)
  (discrete-select
   (1 1/6) (2 1/6) (3 1/6)
   (4 1/6) (5 1/6) (6 1/6)))

(let ((num (+ (roll-die) (roll-die))))
  (observe! (> num 9))
  num)
\end{verbatim}
\begin{center}
\Huge Querying?  Modularity?
\end{center}
\end{frame}

\subsection{The Answer}

\begin{frame}
\begin{center}
\Huge Answer:
\end{center}
\end{frame}

\begin{frame}
\begin{center}
\Huge Both!
\end{center}
\end{frame}

\begin{frame}[fragile=singleslide]
\begin{verbatim}
(define (roll-die)
  (discrete-select
   (1 1/6) (2 1/6) (3 1/6)
   (4 1/6) (5 1/6) (6 1/6)))

(stochastic-thunk->distribution
 (lambda ()
   (let ((num (+ (roll-die) (roll-die))))
     (observe! (> num 9))
     num)))
\end{verbatim}
\end{frame}

\section*{Contributions}

\begin{frame}

\begin{center}
{\Huge Contributions}
\end{center}
\large
  \begin{itemize}
  \item Representation: \alert{Lazy Streams}
  \item API: \alert{Stochastic Functions} AND \alert{Explicit Objects}
  \end{itemize}
  
\end{frame}

\begin{frame}
\begin{center}
\Huge Another example:
\end{center}
\end{frame}

\begin{frame}[fragile=singleslide]

\begin{center}
{\Huge Flipping Coins} \\
The easy way
\end{center}

\begin{verbatim}
(define (num-flips-until-tail)
  (discrete-select
   (0 1/2)
   ((+ 1 (num-flips-until-tail)) 1/2)))

(stochastic-thunk->distribution
 num-flips-until-tail)
\end{verbatim}
\end{frame}

\begin{frame}[fragile=singleslide]

\begin{center}
{\Huge Flipping Coins} \\
The hard way
\end{center}

\begin{verbatim}
(define (coin-flipping-distribution)
  (dependent-product
   (make-discrete-distribution
    '(tails 1/2) '(heads 1/2))
   (lambda (symbol)
     (if (eq? symbol 'tails)
         (make-discrete-distribution '(0 1))
         (coin-flipping-distribution)))
   (lambda (first-flip num-further-flips)
     (if (eq? first-flip 'tails)
         0
         (+ 1 num-further-flips)))))
\end{verbatim}
\end{frame}


\end{document}


